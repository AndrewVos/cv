% Original author:
% Trey Hunner (http://www.treyhunner.com/)

\documentclass{resume}

\usepackage[left=0.75in,top=0.6in,right=0.75in,bottom=0.6in]{geometry}

\usepackage{hyperref}

\name{Andrew Vos}
\address{http://www.andrewvos.com}
\address{(079)~$\cdot$~808~$\cdot$~57010 \\ work@andrewvos.com}

\begin{document}
  \begin{rSection}{Technical Strengths}
    \begin{tabular}{ @{} >{\bfseries}l @{\hspace{6ex}} l }
    Languages & Ruby, C\#, JavaScript, Java, Python, C, C++, PHP \\
    Frameworks & Sinatra, Rails, Rack, Web Forms, Windows Forms, ASP.NET MVC \\
    Databases & MongoDB, MSSQL, MySql, Oracle \\
    Version Control & Git, Mercurial, Subversion, TFS \\
    Provisioning & Vagrant, Puppet, Chef
    \end{tabular}
  \end{rSection}

  \begin{rSection}{Code}
    \begin{tabular}{ @{} >{\bfseries}l @{\hspace{6ex}} l }
    GitHub & \url{http://www.github.com/AndrewVos} \\
    StackOverflow & \url{http://stackoverflow.com/users/128167/andrewvos} \\
    \end{tabular}
  \end{rSection}

  \begin{rSection}{Work Experience}
    \begin{rSubsection}{Sky}{24th September 2012 - Present}{Developer In Test}{London}
      \item Worked across three different teams introducing BDD using Cucumber and Ruby.
      \item Helped with the recruitment of a large number of staff.
      \item Wrote software to enable functional testing of Android Applications (\url{https://github.com/AndrewVos/acouchi}).
    \end{rSubsection}

    \begin{rSubsection}{BBC News}{30th August 2011 - 20th August 2012}{Developer In Test}{London}
      \item Worked with the Responsive News team to help teach the use of BDD/TDD.
      \item Helped with recruitment of contractors and permanent developers.
      \item We redesigned the BBC News site using responsive design techniques to allow the site to work on a large range of mobile phones.
      \item Set up an automated load testing build configuration using Tsung. This build runs a few times every night and allows us to find out if any code changes during the day have negatively impacted our project.
      \item Designed a vagrant / puppet configuration to automatically provision sandbox environments for developers to run their code on.
      \item Trained testers in the use of Capybara so that they could automate some of their manual tests.
      \item Ran various code dojos to help developers learn Ruby, BDD and TDD.
      \item Helped the team move over from Subversion to Git.
      \item Wrote rake tasks to automatically deploy our code to test environments after a successful build.
      \item Worked on a REST api to list news stories by location for the News Summer Labs project.
    \end{rSubsection}

    \begin{rSubsection}{ITV}{14th June 2010 - 29th August 2011}{.NET / Ruby Developer}{London}
      \item Worked on a team writing MRSS integration to allow the BBC to link to ITV videos.
      \item Worked on a team that wrote a data collection API to allow ITV to display more customizable competitions.
      \item Worked with a team of ThoughtWorks developers to bring continuous delivery to our team and also improve coding standards throughout the team.
      \item Moved all developers over to using Moq as a mocking tool, and NUnit as a unit testing framework.
      \item Helped move our developers over to Mercurial from TFS, and providing training/support during the process.
      \item Setup of a TeamCity CI server with a small build server farm. We also set up multiple build status monitors to show the state of the build and also the state of our various staging environments.
      \item One-click deployment for all of our staging/production environments (production is still in progress).
      \item Automation of the entire build process.
      \item Training developers in the use of Cucumber, Capybara and Ruby and promoting Acceptance Testing as part of the development process.
      \item Worked on the fast lane team to deliver important bug fixes directly to production.
      \item Worked on a number of scripts to improve the process and remove the element of human error.
      \item Helped deliver the iOS piece which involved working with Akamai using their HLS streaming product to enable people with iOS devices to view streaming adaptive video.
    \end{rSubsection}

    \begin{rSubsection}{dotCommerce}{24th August 2009 - 13th June 2010}{.NET Developer}{London}
      \item Integration of our framework into pre-sliced sites and the coding of any bespoke functionality for these sites.
    \end{rSubsection}

    \begin{rSubsection}{eForte Limited}{5th March 2009 - 24th August 2009}{PHP Developer}{London}
      \item Small agency
      \item Contract web development
    \end{rSubsection}

    \begin{rSubsection}{WickedOrange}{2008 - 2009}{.NET Developer}{Cape Town}
      \item Freelance web design and development
    \end{rSubsection}

    \begin{rSubsection}{LandscaperPro}{2005 - 2008}{.NET Developer}{Cape Town}
      \item Designed and wrote landscape design software, which we sold to South African gardening retailers.
    \end{rSubsection}
  \end{rSection}

  \begin{rSection}{Open Source}
    \begin{rSubsection}{acouchi}{\url{https://github.com/AndrewVos/acouchi}}{}{}
      \item Functional testing for Android applications
      \item https://github.com/AndrewVos/acouchi
    \end{rSubsection}

    \begin{rSubsection}{Wally}{\url{https://github.com/BBC-News/Wally}}{}{}
      \item A competitor to the popular relishapp.com. This version is open source and hostable. Wally is used at the BBC and other companies.
      \item https://github.com/BBC-News/Wally
    \end{rSubsection}

    \begin{rSubsection}{Spassky}{\url{https://github.com/BBC/spassky}}{}{}
      \item A test framework used to automate the process of running JS unit tests on multiple devices (mostly mobile phones). Used at the BBC to automate testing on mobile phones.
      \item https://github.com/BBC/spassky
    \end{rSubsection}

    \begin{rSubsection}{Coypu}{\url{https://github.com/ITV/Coypu}}{}{}
      \item A .NET wrapper we wrote at ITV which is meant to be used with SpecFlow to automate the browser.
      \item https://github.com/ITV/Coypu
    \end{rSubsection}

    \begin{rSubsection}{rstat.us}{\url{http://rstat.us}}{}{}
      \item rstat.us - An open source Twitter replacement that started being worked on during Twitter’s recent Terms of Service changes.
      \item http://rstat.us/
    \end{rSubsection}
  \end{rSection}

  \begin{rSection}{Smaller Projects}
    \item https://github.com/AndrewVos/docu
    \item https://github.com/AndrewVos/metherd-missing
    \item https://github.com/sthulbourn/anmo
    \item https://github.com/AndrewVos/arduino-tools
    \item https://github.com/AndrewVos/komainu
    \item https://github.com/AndrewVos/oni
  \end{rSection}
\end{document}
